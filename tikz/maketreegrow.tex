
\section{子路径操作介绍(Introduction to the Child Operation)}
	1、语法:
	\begin{lstlisting}[basicstyle=\footnotesize\ttfamily,numbers=left,numberstyle=\tiny, numbersep=9pt, frame=single, language=tex]
\path . . . child[hoptionsi]foreach<variables>in{<values>}{<child path>} . . . ;
	\end{lstlisting}

	“child”操作位于一个完整的node操作之后,或其它的child之后

\section{子路径与子节点(Child Paths and Child Nodes)}
	根节点的每一个子节点(路径)都被自动插入到指定的位置(插入的位置规则,在本章的第五节)

\section{命名子节点(Naming Child Nodes)}


\section{树或子节点路径选项说明(Specifying Options for Trees and Children)}
	
	下面是指定选项的规则集:
	\begin{enumerate}
		\item Options for the whole tree are given before the root node.
		\item Options for the root node are given directly to the node operation of the root.
		\item Options for all children can be given between the root node and the first child.
		\item Options applying to a specific child path are given as options to the child operation.
		\item Options applying to the node of a child, but not to the whole child path, are given as options to the
node command inside the hchild pathi.
	\end{enumerate}
	
	\begin{lstlisting}[basicstyle=\footnotesize\ttfamily,numbers=left,numberstyle=\tiny, numbersep=9pt, frame=single, language=tex]
\begin{tikzpicture}
\scoped
[...] % Options apply to the whole tree
\node[...] {root} % Options apply to the root node only
[...] % Options apply to all children
child[...] % Options apply to this child and all its children
{
node[...] {} % Options apply to the child node only
...
}
child[...] % Options apply to this child and all its children
;
\end{tikzpicture}
	\end{lstlisting}

	影响子节点的样式选项有

	\begin{tikzpicture}[align=center, grow'=right, level distance=1.5cm, sibling distance=2cm, 
	edge from parent/.style={very thick, draw},
	edge from parent path={(\tikzparentnode) to (\tikzchildnode.west)},
	every child node/.style={draw,burntorange, top color=white, bottom color=orange!30, anchor=west, rectangle, rounded corners=3pt},
	notestyle/.style={above right=1cm, callout absolute pointer={(#1.north east)}, ellipse callout,  top color=white, bottom color=lavander!90, draw=violet, font=\tiny},
	]
		\node[draw, rectangle, rounded corners=3pt, top color=white, bottom color=lavander!90, violet]{样式选项}
			child[bend left=25]{node(a){every child}}
			child{node(b){every child node}}
			child{node(c){level=number}}
			child{node(d){level number}};
		\draw (a)node[notestyle={a}]{指定每一个\\子路径的样式};
		\draw (b)node[notestyle={b}]{指定每一个\\子路径中每一个节点的样式};
		\draw (c)node[notestyle={c}]{指定某一层\\或所有层的样式};
		\draw (d)node[notestyle={d}]{指定某一指定层\\的样式};
	\end{tikzpicture}



\section{放置子节点(Placeing Child Nodes)}
	\subsection{基本概念(Basic Idea)}
	\subsection{默认增长函数(Default Growth Function)}
		\begin{tikzpicture}[
		align=center,
		grow'=right,
		edge from parent path={(\tikzparentnode) to(\tikzchildnode.west)},
		level 1/.style={level distance=1.5cm, every node/.style={rectangle, draw, anchor=west}},
		level 2/.style={
			every node/.style={fill=none,anchor=south west},
			level distance=1cm,
			sibling distance=0.6cm
		}]
			\node[draw, cloud] (a) {样式\\选项}
				child{node{level distance}}
				child{node{sibling distance}}
				child{node{grow'}}
				child[sibling distance=3cm]{node{grow}
					child{node{up}}
					child{node{down}}
					child{node{left}}
					child{node{right}}
					child{node{north}}
					child{node{south}}
					child{node{west}}
					child{node{east}}
					child{node{north east}}
					child{node{north west}}
					child{node{south east}}
					child{node{south west}}
					child{node{south west=-135}}
				};
		\end{tikzpicture}
	\subsection{缺失子节点(Missing Children)}
	\subsection{定制增长功能(Custom Growth Functions)}
		\begin{tikzpicture}[
		grow'=right, 
		level distance=3cm, 
		every node/.style={draw, rectangle, anchor=west}]
			\node[cloud] (a) {选项}
				child{node{growth parent}}
				child{node{growth function}};
		\end{tikzpicture}
\section{来自父节点的边(Edges From The Parent Node)}
