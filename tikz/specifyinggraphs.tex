在TikZ中有两种方法生成树:1、使用“graph”路径生成树;2、使用“child”路径生成树,两种方法各有各的优势。
\section{grahp生成}
	1、当要使用\verb"\graph"命令时,需要导入\verb"\usetikzlibrary{graphs}"否则会出现如下错误:
	\begin{lstlisting}[basicstyle=\footnotesize\ttfamily,numbers=left,numberstyle=\tiny, numbersep=9pt, frame=single]
! Package tikz Error: You need to say \usetikzlibrary{graphs} in order to use t
he graph syntax.
	\end{lstlisting}

	2、当要使用\verb"\usegdlibrary"时,要先导入\verb"\usetikzlibrary{graphdrawing}"库

	\begin{lstlisting}[basicstyle=\footnotesize\ttfamily,numbers=left,numberstyle=\tiny, numbersep=9pt, frame=single]
! Undefined control sequence.
l.4 \usegdlibrarysb
               {routing, layered}
	\end{lstlisting}
	
	3、我的理解是\verb"\usegdlibrary{trees}"是图\verb"\graph"绘制的一种布局